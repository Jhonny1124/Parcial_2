\documentclass{article}
\usepackage[utf8]{inputenc}
\usepackage[spanish]{babel}
\usepackage{listings}
\usepackage{graphicx}
\graphicspath{ {images/} }
\usepackage{cite}


\begin{document}
\begin{titlepage}
    \begin{center}
        \vspace*{1cm} 
            
        \Huge
        \textbf{Análisis y Diseño}
            
        \vspace{0.5cm}
        \LARGE
        Informe 
            
        \vspace{1.5cm}
            
        \textbf{Jhonny Alejandro Ortiz Osorio}
        
        \textbf{C.C: 1001015092}
        
        \vfill
            
        \vspace{0.8cm}
            
        \Large
        Departamento de Ingeniería Electrónica y Telecomunicaciones\\
        Universidad de Antioquia\\
        Medellín\\
        Septiembre de 2021
            
    \end{center}
\end{titlepage}

\tableofcontents
\newpage
\section{Análisis del problema}\label{intro}
Para empezar la solución del problema, es importante entender todos los componentes que se deben de integrar tanto para la codificación como para la parte eléctrica en Tinkercad.

El primer paso es crear un programa en el cual se pueda leer una imagen y sacar toda su información de RGB. Después de tener dicha información el programa debe de ajustar el tamaño de la imagen al tamaño del panel LED que se va a tener disponible (de al menos 8x8 leds). Por último, se guarda toda la información de la imagen, ya ajustada a su nuevo tamaño en un archivo.

Con el archivo que se obtuvo del programa se va a poder visualizar la imagen en el panal LED en la plataforma Tinkercad.

\section{Esquema de tareas}\label{intro}
El desafío va a estar dividido en cuatro tareas para poder hacer de forma más practica el trabajo.
\subsection{Lectura y escritura de imagenes}
La primera tarea consiste en conseguir leer una imagen, con la ayuda de la clase QImagen, y sacar toda la información de sus colores en formato RGB, después de esto se hará una prueba haciendo uso de la escritura de imágenes para ver que la información si está bien extraída.
\subsection{Submuestreo y Sobremuestreo}
Esta parte de la solución del programa consiste en crear unos fragmentos de programa, ya sean funciones o clases, donde se realice el submuestreo o sobremuestreo de la imagen, que consiste en ajustar la imagen a un tamaño ya establecido.
\subsection{Guardar la información}
Después de extraer toda la información de la imagen y procesarla para tener el tamaño que se desea, esta información será guardada en un archivo para después utilizarla en el panel de led`s.
\subsection{implementacion Arduino}
Esta ultima tarea consiste en implementar un programa en la plataforma Tinkercad en el cual se pueda utilizar toda la información antes procesada para mostrar la imagen en un panel de led`s.
\end{document}